\documentclass[tikz, border=0.5mm,xcolor=dvipsnames]{standalone}
\usepackage[T1]{fontenc}
\usepackage{mathpazo}
\usepackage[scaled]{beramono}
\usetikzlibrary{positioning}


\begin{document}
	\begin{tikzpicture}[on grid,node distance=0.6cm]
		
		%\draw[help lines,step=5mm,gray!20] (-1,-1) grid (4,4);
		
		\definecolor{ForestGreen}{RGB}{34,139,34}
		\definecolor{LightLightGray}{rgb}{0.9,0.9,0.9}
		
		\pgfmathsetmacro{\pinDist}{0.1}
		\pgfmathsetmacro{\pinNum}{20}
		
		\tikzset{
			piBigConn/.style={
				draw,
				thick,
				fill=LightLightGray,
				minimum size=0.5 cm,
				inner sep=0pt,
				outer sep=0pt,
			},
			piSmallConn/.style={
				draw,
				thick,
				fill=LightLightGray,
				minimum width=0.15 cm,
				minimum height=0.3 cm,
				inner sep=0pt,
				outer sep=0pt,
			},
			piChip/.style={
				draw,
				thick,
				fill=LightLightGray,
				minimum size=0.6 cm,
				inner sep=0pt,
				outer sep=0pt,
				path picture={
					\node[draw,thin,minimum size=0.5cm,font=\tiny\ttfamily] {Wi-Fi};
				};
			},
		}	
	
		
		% Body
		\draw[black, thick, fill=ForestGreen] (0,0) rectangle (2,3);
		
		% Usb and ethernet
		\node[piBigConn] (eth) at (0.4,2.75) {};
		\node[piBigConn, right=of eth] (usb1) {};
		\node[piBigConn, right=of usb1] (usb2) {};
		
		% Connector
		\foreach \i in {1,...,\pinNum} {
			\filldraw ++(0.2,\i*\pinDist) circle (0.02);
			\filldraw ++(0.1,\i*\pinDist) circle (0.02);
		}
		
		% Video and power connectors
		\node[piSmallConn] (power) at (2-0.15/2,0.5) {};
		\node[piSmallConn,above=of power] (hdmi1) {};
		\node[piSmallConn,above=of hdmi1] (hdmi2) {};
		
		% Wi-Fi
		\node[piChip] (wifi) at (0.7,0.4) {};
		
		
		% Logo
		\node at (1,1.5) {\includegraphics[width=0.6cm]{logo_raspberry.pdf}};
		
		
	\end{tikzpicture}
\end{document}
